\nonstopmode{}
\documentclass[letterpaper]{book}
\usepackage[times,inconsolata,hyper]{Rd}
\usepackage{makeidx}
\usepackage[utf8,latin1]{inputenc}
% \usepackage{graphicx} % @USE GRAPHICX@
\makeindex{}
\begin{document}
\chapter*{}
\begin{center}
{\textbf{\huge Package `ExpDesignPkg'}}
\par\bigskip{\large \today}
\end{center}
\begin{description}
\raggedright{}
\item[Type]\AsIs{Package}
\item[Title]\AsIs{Experimental design}
\item[Version]\AsIs{1.5}
\item[Date]\AsIs{2015-12-07}
\item[Author]\AsIs{Georgia Tsiliki}
\item[Maintainer]\AsIs{Georgia Tsiliki }\email{gtsiliki@central.ntua.gr}\AsIs{}
\item[Depends]\AsIs{RCurl, pmml, jsonlite, AlgDesign, pls}
\item[Description]\AsIs{This package performs experimental design by employing algorithmic design R routines from the AlgDesign package. The Brandmaier et al method is also employed.}
\item[License]\AsIs{GPL-2}
\item[NeedsCompilation]\AsIs{no}
\end{description}
\Rdcontents{\R{} topics documented:}
\inputencoding{utf8}
\HeaderA{ExpDesignPkg-package}{Experimental design}{ExpDesignPkg.Rdash.package}
\aliasA{ExpDesignPkg}{ExpDesignPkg-package}{ExpDesignPkg}
\keyword{package}{ExpDesignPkg-package}
%
\begin{Description}\relax
Calculates an exact or approximate algorithmic design for one of three criteria, using Federov's exchange algorithm from AlgDesign package
\end{Description}
%
\begin{Details}\relax

The DESCRIPTION file:

\Tabular{ll}{
Package: & ExpDesignPkg\\{}
Type: & Package\\{}
Title: & Experimental design\\{}
Version: & 1.5\\{}
Date: & 2015-12-07\\{}
Author: & Georgia Tsiliki\\{}
Maintainer: & Georgia Tsiliki <gtsiliki@central.ntua.gr>\\{}
Depends: & RCurl, pmml, jsonlite, AlgDesign, pls\\{}
Description: & This package performs experimental design by employing algorithmic design R routines from the AlgDesign package. The Brandmaier et al method is also employed. \\{}
License: & GPL-2\\{}
}

Index of help topics:
\begin{alltt}
ExpDesignPkg-package    Experimental design
dat1                    A sample data object
dat1i                   Information for experimental design function
                        suggest.trials.xy
dat1m                   Serialized experimental design model file
dat1p                   A sample data object
dat2i                   Information for experimental design function
                        suggest.trials.noxy
dat2m                   Serialized factorial experimental design model
                        file
exp.design.noxy         Experimental design function for (full)
                        factorial designs
exp.design.xy           Experimental design function with X and/or y
                        values
r2.adj.funct            Adjusted R2 function
r2.funct                R2 function
suggest.trials.noxy     Returns suggested trials for a factorial design
suggest.trials.xy       Returns suggested trials when data are
                        available
\end{alltt}

The most important functions of the package are exp.design.funct.xy and exp.design.funct.noxy. They produce a a suggested trials vector/ matrix respectively serialized for suggested.trials.xy and suggested.trials.noxy to handle. 
\end{Details}
%
\begin{Author}\relax
Georgia Tsiliki

Maintainer: Georgia Tsiliki <gtsiliki@central.ntua.gr>
\end{Author}
%
\begin{References}\relax
Help files of AlgDesign
\end{References}
%
\begin{Examples}
\begin{ExampleCode}

data("dat1")

predF<- "https://apps.ideaconsult.net/enmtest/property/TOX/UNKNOWN_TOXICITY_SECTION/Net+cell+association/8058CA554E48268ECBA8C98A55356854F413673B/3ed642f9-1b42-387a-9966-dea5b91e5f8a"

required.param<- list(nTrials=c(11),criterion='D',form='linear',r2.threshold=0.9)

exp.example<- exp.design.xy(dat1,predF,required.param) 

\end{ExampleCode}
\end{Examples}
\inputencoding{utf8}
\HeaderA{dat1}{A sample data object}{dat1}
\keyword{datasets}{dat1}
%
\begin{Description}\relax
The dataset for this test is a data frame
\end{Description}
%
\begin{Usage}
\begin{verbatim}
data("dat1")
\end{verbatim}
\end{Usage}
%
\begin{Format}
A list of two objects
\begin{description}

\item[datasetURI] a character vector- ambit data set uri
\item[dataEntry] a data frame containing two columns: compound and values. Compound is a character vector with all compound ambit uris, and values is a data frame with all numberic values of the protein corona data set (compounds by features). One of the columns is the dependent variable (Net.cell.association) which has some null values- the experimental design algorithm will select some/all of them for next suggested trials.

\end{description}

\end{Format}
%
\begin{Details}\relax
There are no more details
\end{Details}
%
\begin{Source}\relax
The source of this function is in the
\end{Source}
%
\begin{References}\relax
There are no references
\end{References}
%
\begin{Examples}
\begin{ExampleCode}
data(dat1)
## maybe str(dat1) ; plot(dat1) ...
\end{ExampleCode}
\end{Examples}
\inputencoding{utf8}
\HeaderA{dat1i}{Information for experimental design function suggest.trials.xy}{dat1i}
\keyword{datasetsi}{dat1i}
%
\begin{Description}\relax
A list with information for experimental design function suggest.trials.xy
\end{Description}
%
\begin{Usage}
\begin{verbatim}
data("dat1i")
\end{verbatim}
\end{Usage}
%
\begin{Format}
A list with 8 objects:
\begin{description}

\item[\code{design}] a numeric matrix with the suggested design for the data
\item[\code{selected.rows}] the selected rows for all the available combinations of the parameters.
\item[\code{norm.var}] the normalized variance. This is the Ge value from optFederov(): The minimax normalized variance over X, expressed as an efficiency with respect to the optimal approximate theory design. Optimal value is 1.
\item[\code{confounding.effect}] the diagonality of the design, excluding the constant if any, from eval.design() function. The optimal number is 1.
\item[\code{r.squared}] the r2 value for the data supplied
\item[\code{adj.r.squared}] the adjusted r2 value for the data supplied.
\item[\code{verbal.notes}] remarks commenting on Ge and diagonality.

\end{description}

\end{Format}
%
\begin{Details}\relax
Example dataset to suggest trials together with dat1p, dat1m
\end{Details}
%
\begin{Source}\relax
The source of this function is in the
\end{Source}
%
\begin{References}\relax
There are no references
\end{References}
%
\begin{Examples}
\begin{ExampleCode}
data(dat1i)
## maybe str(dat1i) ; plot(dat1i) ...
\end{ExampleCode}
\end{Examples}
\inputencoding{utf8}
\HeaderA{dat1m}{Serialized experimental design model file }{dat1m}
\keyword{datasetsm}{dat1m}
%
\begin{Description}\relax
A character string for a serialized experimental design model, i.e. a list including one vectorindicating which are the next trials that should be conducted.
\end{Description}
%
\begin{Usage}
\begin{verbatim}
data("dat1m")
\end{verbatim}
\end{Usage}
%
\begin{Format}
A character string
\end{Format}
%
\begin{Details}\relax
Example experimental design model based on dat1
\end{Details}
%
\begin{Source}\relax
The source of this function is in the
\end{Source}
%
\begin{References}\relax
There are no references
\end{References}
%
\begin{Examples}
\begin{ExampleCode}
data(dat1m)
## maybe str(dat1m) ; plot(dat1m) ...
\end{ExampleCode}
\end{Examples}
\inputencoding{utf8}
\HeaderA{dat1p}{A sample data object}{dat1p}
\keyword{datasetsp}{dat1p}
%
\begin{Description}\relax
The dataset for this test is a data frame
\end{Description}
%
\begin{Usage}
\begin{verbatim}
data("dat1p")
\end{verbatim}
\end{Usage}
%
\begin{Format}
A list of two objects
\begin{description}

\item[datasetURI] a character vector- ambit data set uri
\item[dataEntry] a data frame containing two columns: compound and values. Exactly the same as dat1.

\end{description}

\end{Format}
%
\begin{Details}\relax
Data set for prediction with dat1m, although data are not used by the function
\end{Details}
%
\begin{Source}\relax
The source of this function is in the
\end{Source}
%
\begin{References}\relax
There are no references
\end{References}
%
\begin{Examples}
\begin{ExampleCode}
data(dat1p)
## maybe str(dat1p) ; plot(dat1p) ...
\end{ExampleCode}
\end{Examples}
\inputencoding{utf8}
\HeaderA{dat2i}{Information for experimental design function suggest.trials.noxy}{dat2i}
\keyword{datasets2i}{dat2i}
%
\begin{Description}\relax
A list with information for experimental design function suggest.trials.noxy
\end{Description}
%
\begin{Usage}
\begin{verbatim}
data("dat2i")
\end{verbatim}
\end{Usage}
%
\begin{Format}
A list with 6 objects:
\begin{description}

\item[\code{design}] a numeric matrix with the suggested design for the data
\item[\code{selected.rows}] the selected rows for all the available combinations of the parameters.
\item[\code{norm.var}] the normalized variance. This is the Ge value from optFederov(): The minimax normalized variance over X, expressed as an efficiency with respect to the optimal approximate theory design. Optimal value is 1.
\item[\code{confounding.effect}] the diagonality of the design, excluding the constant if any, from eval.design() function. The optimal number is 1.
\item[\code{verbal.notes}] remarks commenting on Ge and diagonality.

\end{description}

\end{Format}
%
\begin{Details}\relax
Example dataset to suggest trials together with dat2m
\end{Details}
%
\begin{Source}\relax
The source of this function is in the
\end{Source}
%
\begin{References}\relax
There are no references
\end{References}
%
\begin{Examples}
\begin{ExampleCode}
data(dat2i)
## maybe str(dat2i) ; plot(dat2i) ...
\end{ExampleCode}
\end{Examples}
\inputencoding{utf8}
\HeaderA{dat2m}{Serialized factorial experimental design model file }{dat2m}
\keyword{datasets2m}{dat2m}
%
\begin{Description}\relax
A character string for a serialized factorial design, i.e. a list including one vector indicating which are the trials that should be conducted.
\end{Description}
%
\begin{Usage}
\begin{verbatim}
data("dat2m")
\end{verbatim}
\end{Usage}
%
\begin{Format}
A character string
\end{Format}
%
\begin{Details}\relax
Example experimental design model produced by exp.design.noxy function
\end{Details}
%
\begin{References}\relax
There are no references
\end{References}
%
\begin{Examples}
\begin{ExampleCode}
data(dat2m)
## maybe str(dat2m) ; plot(dat2m) ...
\end{ExampleCode}
\end{Examples}
\inputencoding{utf8}
\HeaderA{exp.design.noxy}{Experimental design function for (full) factorial designs}{exp.design.noxy}
\keyword{expDesignNoXY}{exp.design.noxy}
%
\begin{Description}\relax
Calculates an exact or approximate algorithmic design for one of three criteria, using Federov's exchange algorithm from AlgDesign package. The user needs to specify the number of variables and their levels, then a design matrix with suggested trials is produced. Please note in this case, no X or Y values are provided.
\end{Description}
%
\begin{Usage}
\begin{verbatim}
exp.design.noxy(dataset, predictionFeature, parameters)
\end{verbatim}
\end{Usage}
%
\begin{Arguments}
\begin{ldescription}
\item[\code{dataset}] list of 2 objects, datasetURI and dataEntry - NOT required 

\item[\code{predictionFeature}]  character string specifying which is the prediction feature in dataEntry - NOT required

\item[\code{parameters}] A list with 7 objects: levels, nVars (number of variables), factors (which are the factor variables), varNames (variables' names), nTrials (the number of suggested trials, if 0 then an estimated number is suggested), criterion ('A', 'I', 'D'), form of the design ('linear','quad','cubic','cubicS')

\end{ldescription}
\end{Arguments}
%
\begin{Details}\relax
No details required
\end{Details}
%
\begin{Value}
A List 
\begin{ldescription}
\item[\code{rawModel}]  A serialized numeric matrix indicating the experimental design for the various variables and their levels.
\item[\code{pmmlModel}]  A pmml object - now empty
\item[\code{independentFeatures}]  A list with the names of the variables as given by the user.
\item[\code{predictedFeatures}] A character vector with names for the suggested trials
\item[\code{additionalInfo}]  A list including the following: design (The design suggested), selected.rows (The rows (nanoparticles) suggested for new trials), norm.values (The minimax normalized variance over X, expressed as an efficiency with respect to the optimal approximate theory design.), confounding.effect (The diagonality of the design, excluding the constant, if any.), verbal.notes (Verbal notes to comment on norm.values and confounding.effect), predictedFeatures (The character string 'suggestedTrials' to indicate the new vector created including all experimental design memberships).
\end{ldescription}
\end{Value}
%
\begin{Note}\relax
 No notes for this function 
\end{Note}
%
\begin{Author}\relax
Georgia Tsiliki
\end{Author}
%
\begin{References}\relax
The help file of blockcluster package 
\end{References}
%
\begin{Examples}
\begin{ExampleCode}
##---- Should be DIRECTLY executable !! ----
##-- ==>  Define data, use random,
##--	or do  help(data=index)  for the standard data sets.


required.param<-  list(levels=3, nVars=3, factors='null', varNames=c('a','b','c'),nTrials=10,criterion='D',form='linear')


exp.example<- exp.design.noxy(null,null,required.param) 
\end{ExampleCode}
\end{Examples}
\inputencoding{utf8}
\HeaderA{exp.design.xy}{Experimental design function with X and/or y values }{exp.design.xy}
\keyword{expDesignXY}{exp.design.xy}
%
\begin{Description}\relax
Calculates an exact or approximate algorithmic design for one of three criteria, using Federov's exchange algorithm from AlgDesign package
\end{Description}
%
\begin{Usage}
\begin{verbatim}
exp.design.xy(dataset, predictionFeature, parameters)
\end{verbatim}
\end{Usage}
%
\begin{Arguments}
\begin{ldescription}
\item[\code{dataset}] list of 2 objects, datasetURI:= character sring, code name of dataset, dataEntry:= data frame with 2 columns 

\item[\code{predictionFeature}]  character string specifying which is the prediction feature in dataEntry 

\item[\code{parameters}] list with parameter values for experimental design. 5 objects should be included, i.e. nTrials a numeric value indicating number of trials suggested, if 0 then an estimated number is suggested, criterion a character value to indicate which optimal deisgn to apply (possible values are  'D', 'A', 'I'), form a string indicating the formula of the deisgn (possible formulas are 'linear','quad','cubic','cubicS'), r2.threshold a numeric value indicating the r2 threshold value (If the data supplied provides
r2 value greater than the threshold value, a warning message is returned.). The 5th object is only needed in case the dependent variable is not available yet, then a new parameter should be added, called 'newY' which is a character string with the name of the new dependent feature.

\end{ldescription}
\end{Arguments}
%
\begin{Details}\relax
No details required 
\end{Details}
%
\begin{Value}
A List 
\begin{ldescription}
\item[\code{rawModel}]  A serialized numeric vector indicating the experimental design memberships of nanoparticles in the data.
\item[\code{pmmlModel}]  A pmml object - now empty
\item[\code{independentFeatures}]  A list with Ambit names for all genes/ proteins features included in the model 
\item[\code{predictedFeatures}] A character vector with names for the suggested trials
\item[\code{additionalInfo}]  A list including the following: design (The design suggested), selected.rows (The rows (nanoparticles) suggested for new trials), norm.values (The minimax normalized variance over X, expressed as an efficiency with respect to the optimal approximate theory design.), confounding.effect (The diagonality of the design, excluding the constant, if any.), r.squared (when y is given), adj.r.squared (when y is given), verbal.notes (Verbal notes to comment on norm.values and confounding.effect), predictedFeatures (The character string 'suggestedTrials' and 'newY' to indicate the new vectors created including all experimental design memberships).
\end{ldescription}
\end{Value}
%
\begin{Note}\relax
 No notes for this function 
\end{Note}
%
\begin{Author}\relax
Georgia Tsiliki
\end{Author}
%
\begin{References}\relax
The help file of blockcluster package 
\end{References}
%
\begin{Examples}
\begin{ExampleCode}
##---- Should be DIRECTLY executable !! ----
##-- ==>  Define data, use random,
##--  or do  help(data=index)  for the standard data sets.

data("dat1")

predF<- "https://apps.ideaconsult.net/enmtest/property/TOX/UNKNOWN_TOXICITY_SECTION/Net+cell+association/8058CA554E48268ECBA8C98A55356854F413673B/3ed642f9-1b42-387a-9966-dea5b91e5f8a"

required.param<- list(nTrials=c(11),criterion='D',form='linear',r2.threshold=0.9)

exp.example<- exp.design.xy(dat1,predF,required.param) 

\end{ExampleCode}
\end{Examples}
\inputencoding{utf8}
\HeaderA{r2.adj.funct}{ Adjusted R2 function }{r2.adj.funct}
\keyword{r2adj}{r2.adj.funct}
%
\begin{Description}\relax
Calculates the adjusted R2 value
\end{Description}
%
\begin{Usage}
\begin{verbatim}
r2.adj.funct(y, y.new, num.pred)
\end{verbatim}
\end{Usage}
%
\begin{Arguments}
\begin{ldescription}
\item[\code{y}]  observed y values 
\item[\code{y.new}]  predicted y values 
\item[\code{num.pred}]  number of parameters in the predicted model
\end{ldescription}
\end{Arguments}
%
\begin{Details}\relax
No details required 
\end{Details}
%
\begin{Value}
 A numeric value for the adjusted coefficient of determination R2 
\end{Value}
%
\begin{Author}\relax
Georgia Tsiliki 
\end{Author}
%
\begin{References}\relax
 Dobson An introduction to linear modelling 
\end{References}
%
\begin{Examples}
\begin{ExampleCode}
##---- Should be DIRECTLY executable !! ----
##-- ==>  Define data, use random,
##--	or do  help(data=index)  for the standard data sets.

r2.adj.funct(1:10,1:10,2)

\end{ExampleCode}
\end{Examples}
\inputencoding{utf8}
\HeaderA{r2.funct}{R2 function}{r2.funct}
\keyword{r2}{r2.funct}
%
\begin{Description}\relax
 Calculates the R2 value 
\end{Description}
%
\begin{Usage}
\begin{verbatim}
 r2.funct(y, y.new) 
\end{verbatim}
\end{Usage}
%
\begin{Arguments}
\begin{ldescription}
\item[\code{y}]  observed y values
\item[\code{y.new}]  predicted y values
\end{ldescription}
\end{Arguments}
%
\begin{Details}\relax
No details required 
\end{Details}
%
\begin{Value}
A numeric value for the coefficient of determination R2 
\end{Value}
%
\begin{Author}\relax
 Georgia Tsiliki 
\end{Author}
%
\begin{References}\relax
Dobson An introduction to linear modelling
\end{References}
%
\begin{Examples}
\begin{ExampleCode}
##---- Should be DIRECTLY executable !! ----
##-- ==>  Define data, use random,
##--	or do  help(data=index)  for the standard data sets.

r2.funct(1:10,1:10)
\end{ExampleCode}
\end{Examples}
\inputencoding{utf8}
\HeaderA{suggest.trials.noxy}{Returns suggested trials for a factorial design}{suggest.trials.noxy}
\keyword{suggestTFF}{suggest.trials.noxy}
%
\begin{Description}\relax
Suggested trials are returned as a list with one object, a numeric matrix with the last column indicating with 1 the suggested trial
\end{Description}
%
\begin{Usage}
\begin{verbatim}
suggest.trials.noxy(dataset, rawModel, additionalInfo)
\end{verbatim}
\end{Usage}
%
\begin{Arguments}
\begin{ldescription}
\item[\code{dataset}]  Data for prediction. A list of two objects: datasetURI (a character string ), dataEntry (a data frame).
\item[\code{rawModel}]  R model serialized (suggested trials for the data matrix supplied in exp.design.funct1). 

\item[\code{additionalInfo}]  Any additional information needed for rawModel. Here the list generated by exp.design.funct.noxy. The list should contain a field named 'predictedFeatures' which should be exactly the same as that returned by exp.design.funct.noxy function. 

\end{ldescription}
\end{Arguments}
%
\begin{Details}\relax
 No further details required 
\end{Details}
%
\begin{Value}
A list of one objected called 'predictions' which is also a list of one cell data-frames each containing the suggested trials for the data tested. This object is a matrix for the suggested trials per variable and level; the last column indicates with 1 the suggested trial
\end{Value}
%
\begin{Note}\relax
 No notes for this function 
\end{Note}
%
\begin{Author}\relax
Georgia Tsiliki
\end{Author}
%
\begin{References}\relax
 No references required 
\end{References}
%
\begin{Examples}
\begin{ExampleCode}
##---- Should be DIRECTLY executable !! ----
##-- ==>  Define data, use random,
##--	or do  help(data=index)  for the standard data sets.


data("dat2m")
data("dat2i")

pred.res<- suggest.trials.noxy(null, dat2m, dat2i) 

\end{ExampleCode}
\end{Examples}
\inputencoding{utf8}
\HeaderA{suggest.trials.xy}{Returns suggested trials when data are available}{suggest.trials.xy}
\keyword{suggestT}{suggest.trials.xy}
%
\begin{Description}\relax
Suggested trials are returned as a list with one object, a binary vector where 1 indicates suggested trial
\end{Description}
%
\begin{Usage}
\begin{verbatim}
suggest.trials.xy(dataset, rawModel, additionalInfo)
\end{verbatim}
\end{Usage}
%
\begin{Arguments}
\begin{ldescription}
\item[\code{dataset}]  Data for prediction. A list of two objects: datasetURI (a character string ), dataEntry (a data frame).
\item[\code{rawModel}]  R model serialized (suggested trials for the data matrix supplied in exp.design.funct1). 

\item[\code{additionalInfo}]  Any additional information needed for rawModel. Here the list generated by exp.design.funct.xy. The list should contain a field named 'predictedFeatures' which should be exactly the same as that returned by exp.design.funct.xy function. 

\end{ldescription}
\end{Arguments}
%
\begin{Details}\relax
 No further details required 
\end{Details}
%
\begin{Value}
A list of one objected called 'predictions' which is also a list of one cell data-frames each containing the suggested trials for the data tested. This object is a binary vector with 1 indicates suggested trial
\end{Value}
%
\begin{Note}\relax
 No notes for this function 
\end{Note}
%
\begin{Author}\relax
Georgia Tsiliki
\end{Author}
%
\begin{References}\relax
 No references required 
\end{References}
%
\begin{Examples}
\begin{ExampleCode}
##---- Should be DIRECTLY executable !! ----
##-- ==>  Define data, use random,
##--  or do  help(data=index)  for the standard data sets.

data("dat1p")
data("dat1m")
data("dat1i")

pred.res<- suggest.trials.xy(dat1p, dat1m, dat1i) 

\end{ExampleCode}
\end{Examples}
\printindex{}
\end{document}
